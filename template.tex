%% start of file `template.tex'.
%% Copyright 2006-2015 Xavier Danaux (xdanaux@gmail.com).
%
% This work may be distributed and/or modified under the
% conditions of the LaTeX Project Public License version 1.3c,
% available at http://www.latex-project.org/lppl/.


\documentclass[11pt,a4paper,sans]{moderncv}        % possible options include font size ('10pt', '11pt' and '12pt'), paper size ('a4paper', 'letterpaper', 'a5paper', 'legalpaper', 'executivepaper' and 'landscape') and font family ('sans' and 'roman')

% moderncv themes
\moderncvstyle{casual}                             % style options are 'casual' (default), 'classic', 'banking', 'oldstyle' and 'fancy'
\moderncvcolor{blue}                               % color options 'black', 'blue' (default), 'burgundy', 'green', 'grey', 'orange', 'purple' and 'red'
%\renewcommand{\familydefault}{\sfdefault}         % to set the default font; use '\sfdefault' for the default sans serif font, '\rmdefault' for the default roman one, or any tex font name
%\nopagenumbers{}                                  % uncomment to suppress automatic page numbering for CVs longer than one page

% character encoding
\usepackage[utf8]{inputenc}                       % if you are not using xelatex ou lualatex, replace by the encoding you are using
%\usepackage{CJKutf8}                              % if you need to use CJK to typeset your resume in Chinese, Japanese or Korean

% adjust the page margins
\usepackage[scale=0.75]{geometry}
%\setlength{\hintscolumnwidth}{3cm}                % if you want to change the width of the column with the dates
%\setlength{\makecvtitlenamewidth}{10cm}           % for the 'classic' style, if you want to force the width allocated to your name and avoid line breaks. be careful though, the length is normally calculated to avoid any overlap with your personal info; use this at your own typographical risks...

% personal data
\name{Michaela}{Greplová}
\title{CV}                               % optional, remove / comment the line if not wanted
\address{Šámalova 208}{60200}{Česká republika}% optional, remove / comment the line if not wanted; the "postcode city" and "country" arguments can be omitted or provided empty
\phone[mobile]{+420~720~332~591}                   % optional, remove / comment the line if not wanted; the optional "type" of the phone can be "mobile" (default), "fixed" or "fax"
%\phone[fixed]{+2~(345)~678~901}
%\phone[fax]{+3~(456)~789~012}
\email{469400@mail.muni.cz}                               % optional, remove / comment the line if not wanted
%\homepage{www.greplovam.com}                         % optional, remove / comment the line if not wanted
%\social[linkedin]{john.doe}                        % optional, remove / comment the line if not wanted
\social[facebook]{Michaela Greplová}                             % optional, remove / comment the line if not wanted
\social[github]{xgreplov}                              % optional, remove / comment the line if not wanted
%\extrainfo{additional information}                 % optional, remove / comment the line if not wanted
\photo[64pt][0.4pt]{picture}                       % optional, remove / comment the line if not wanted; '64pt' is the height the picture must be resized to, 0.4pt is the thickness of the frame around it (put it to 0pt for no frame) and 'picture' is the name of the picture file
%\quote{Some quote}                                 % optional, remove / comment the line if not wanted

% bibliography adjustements (only useful if you make citations in your resume, or print a list of publications using BibTeX)
%   to show numerical labels in the bibliography (default is to show no labels)
\makeatletter\renewcommand*{\bibliographyitemlabel}{\@biblabel{\arabic{enumiv}}}\makeatother
%   to redefine the bibliography heading string ("Publications")
%\renewcommand{\refname}{Articles}

% bibliography with mutiple entries
%\usepackage{multibib}
%\newcites{book,misc}{{Books},{Others}}
%----------------------------------------------------------------------------------
%            content
%----------------------------------------------------------------------------------
\begin{document}
%\begin{CJK*}{UTF8}{gbsn}                          % to typeset your resume in Chinese using CJK
%-----       resume       ---------------------------------------------------------
\makecvtitle

\section{Vzdělání}
\cventry{2011--2017}{Maturita}{Gymnázium Olomouc}{Olomouc}{}{zaměření Anglická sekce}  % arguments 3 to 6 can be left empty
\cventry{2017--2020}{Bakalářský titul}{Masarykova Univerzita}{Brno}{}{obor Aplikovaná Informatika}

\section{Bakalářská práce}
\cvitem{název}{\emph{Modul pro zobrazování formulářů ve formátu FRS v rámci webové aplikace}}
\cvitem{vedoucí práce}{Mgr. Martin Macák}
\cvitem{popis}{FRS je proprietární datový formát popisující vizuální podobu strukturovaných formulářů obsahujících různé datové prvky, který je použitý primárně pro prezentaci jízdních řádů koncovým uživatelům nebo pracovníkům v rámci hromadné dopravy.}

\section{Zkušenosti}
\subsection{Odborné}
\cventry{2017--2019}{Programátor Junior}{FS Software s.r.o.}{Olomouc}{}{Práce na jízdních řádech \newline{}%
%Detailed achievements:%
%\begin{itemize}%
}
\cventry{2017--2019}{Tester}{K.A.P.}{Olomouc}{}{Testování mobilní aplikace\newline{}Testování funkčnosti multiplatformní aplikace}
\subsection{Jiné}
\cventry{2013--2017}{Lektor anličtiny}{Soukromý}{Olomouc}{}{Idividuální i skupinová výuka angličtiny}

\section{Jazyky}
\cvitemwithcomment{Angličtina}{C2}{Možnost doložení certifikátem CAE}
\cvitemwithcomment{Němčina}{B2}{Spíše na komunikační úrovni}

\section{Programovací jazyky}
\cvitem{1}{C\#}
\cvitem{2}{Java}
\cvitem{3}{C++}
\cvitem{4}{C}

% Publications from a BibTeX file without multibib
%  for numerical labels: \renewcommand{\bibliographyitemlabel}{\@biblabel{\arabic{enumiv}}}% CONSIDER MERGING WITH PREAMBLE PART
%  to redefine the heading string ("Publications"): \renewcommand{\refname}{Articles}
\nocite{*}
\bibliographystyle{plain}
\bibliography{publications}                        % 'publications' is the name of a BibTeX file

% Publications from a BibTeX file using the multibib package
%\section{Publications}
%\nocitebook{book1,book2}
%\bibliographystylebook{plain}
%\bibliographybook{publications}                   % 'publications' is the name of a BibTeX file
%\nocitemisc{misc1,misc2,misc3}
%\bibliographystylemisc{plain}
%\bibliographymisc{publications}                   % 'publications' is the name of a BibTeX file

\clearpage

\end{document}


%% end of file `template.tex'.
